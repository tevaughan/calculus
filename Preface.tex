
\chapter{Preface}

In the Summer of the year 2014 since the Incarnation of the Logos,\footnote{%
   Referring to the Logos---the reason behind the existence of all things---as
   {\selectlanguage{greek}a>uto~u} (him), the third verse of the Gospel of John
   in English is, ``It was through him that all things came into being, and
   without him came nothing that has come to be.''
   {\selectlanguage{greek}p'anta di a>uto~u >eg'eneto, ka`i qwr`is a>uto~u
   >eg'eneto o>ud`e <'en <`o g'egonen.} (Original Greek and English translation
   from \texttt{newadvent.org}).  I invoke the Logos here for several reasons.
   \begin{enumerate}
      \item Thinking about the origin of the present book reminds me of origins
         in general.  From the Incarnation of the Logos derives the existence
         of all space, all past, and all future.  Christmas is not the
         \emph{temporal} origin of things, but Christmas is the \emph{logical}
         origin of all physical reality, including the present book. Man cannot
         merely by study of ordinary things in the physical universe be certain
         of its age (or even, as Aquinas pointed out, whether its age be
         \emph{finite}).  Based on assumptions that can be neither proved nor
         disproved, the belief in a young age for the universe can be more or
         less reasonable, but the insistence on everyone's holding such
         assumptions---lest one risk eternal damnation---is wrong.
         St.~Augustine, the great doctor of the Church, himself argued in the
         early 400s A.D.~both for an old Earth and a for a process of evolution
         for the human body; see his {\it De Genesi ad Litteram}.  The entire
         history of the physical world is not so much defined by the conditions
         at the beginning of time as by the conditions at the Incarnation;
         unlike the temporal beginning of the physical universe, the logical
         beginning of all things, including time, is located in a special way
         in the middle of history.
      \item In my experience, the number of the year is too often used without
         any reference to the origin of the temporal coordinate system. We
         should at least occasionally mention what the first year is.
      \item Because the present book promotes modern mathematics and science,
         to begin by mentioning the tradition in which they arose is fitting.
         Doubly fitting is so to begin in the midst of a culture awash in the
         false perception of a substantial conflict between what is proper to
         modern science and what is proper to traditional Christianity.
      \item The Logos is ultimately what science and mathematics are all about.
         The idea of comprehensible, logical, consistent reason behind things
         is the very idea of the Logos. On the one hand, the traditional
         Christian identifies the Logos as the Second Person of the Trinity.
         On the other hand, modern science is squarely directed toward the
         Logos, though in the attempt to connect mathematics to the physical
         universe, science can never grasp the Logos.
   \end{enumerate}
}
I began to write the present book. My hope is that some day it will be used in
a home school, perhaps even for my younger children, if I should finish the
book soon enough.

My mother read Wray G.~Brady's and Maynard J.~Mansfield's {\it Calculus\/}
(1960) when she attended the University of St.~Thomas Aquinas in Houston,
Texas.  She did not see the point of studying the calculus.  I dislike some of
the notation in Brady and Mansfield, but I do like their approach. The authors
develop the calculus by theorems from definitions of the set, the function, and
the limit.  This approach, enabled by mathematical ideas introduced in the 19th
Century, frees the calculus from ambiguities and logical inconsistencies that
for at least 2,500 years plagued mathematical thought on the concept of the
vanishingly small. I shall at least roughly follow Brady and Mansfield, but I
shall use different notation; also, I shall add some material from geometry,
physics, and philosophy.

Because of my positive high-school experience in simultaneously studying the
calculus and calculus-based physics, I have long thought that the calculus
ought ideally to be introduced along with some basic physics.  So I plan to
present both here. The combination might address my mother's dislike for the
absence of any recognizable motivation for the calculus.  The main results of
the calculus, established by Newton and Leibniz more than 200 years before the
rigorous approach of the 19th Century was worked out, are well motivated by
physical intuition.  Newton developed the calculus to serve as the language of
physical theory.  Because the calculus is arguably most naturally intuited in
the context of Newtonian mechanics, I shall develop here along with the
calculus some relevant aspects of the mechanics.

Developing the mathematics in the context of the mechanical theory gives me the
opportunity to make careful distinctions.  In my recent reading of Carl B.
Boyer's {\it The History of the Calculus and Its Conceptual Development}, I
was, particularly in his chapter on mathematics in the ancient world, reminded
of man's perennial and unfortunate tendency to confuse mathematics with nature.
Looking back on the emergence of modern science in the 17th Century, one sees
that there are three relevant domains carefully to be distinguished:
\begin{enumerate}
   \item mathematics,
   \item scientific theory, and
   \item the natural world.
\end{enumerate}

Surveying the long history of mathematics, Boyer notes ``that mathematics is
the study of relationships in general and must not be hampered by any
preconceived notions, derived from sensory perception, of what these
relationships should be.''\footnote{%
   Boyer, {\it The History of the Calculus and Its Conceptual Development},
   p.~13.%
}
The problem is that logic applied to definitions and postulates can lead to
conclusions apparently inconsistent with sense experience. Among the possible
responses to the inconsistency are two extremes, each of which seems wrong:
\begin{itemize}
   \item On the one hand, to throw away a piece of mathematics with intrinsic
      beauty, internal consistency, and practical utility seems wrong.
   \item On the other hand, to disregard what is known by sense experience
      seems wrong.
\end{itemize}
A moderate approach is to admit that while the seemingly obvious definitions
and postulates at the root of a mathematical conclusion do not correspond
exactly to the world of sense experience, there is some utility in seeing how
far any given correspondendence can be taken.  Around the same time as the
development of the rigorous approach to the calculus, and after about 200 years
of consistency with sense experience, the modern scientific theory of Newtonian
mechanics began in the late 19th Century to draw conclusions inconsistent with
sense experience. Already with the advent of Maxwell's equations, there was a
mathematical inconsistency between the Newtonian mechanics and the new
electromagnetic theory; this inconsistency eventually led to Einstein's
development of special relativity. The first observational inconsistency,
though, appeared in measurements of the radiation emitted by an object of a
given temperature.  The Newtonian mechanics predicted a spectral energy
distribution different from what was observed.  The inconsistency arose not
because of any error in the mathematics but because technological advancement
exposed the error of the scientific postulate---the scientific hypothesis---by
which the mathematical is connected to the physical. In this case, the
hypothesis is that the same mathematics describing the motion of objects large
enough directly to be perceived by the senses also describes the motion of
objects too small to be perceived.  This hypothesis, however, led to a failure
adequately to account for what is in fact observed by the senses.  Nature is
imperfectly described by scientific theory, and the imperfection in the theory
lies in the assertion of a particular connection between mathematics and
nature.

Newtonian mechanics and later scientific theories based in one way or another
on the calculus are amazingly good at predicting what the observer of a
scientific experiment will perceive through the senses.  Nevertheless, I intend
in the text at least occasionally to point out that modern science can
\emph{never} determine what in nature lies beyond sense experience. A modern
scientific theory inevitably attempts to connect mathematics to nature at
points that are outside the realm of perception. These are precisely the points
at which we cannot be certain what---if any---mathematics ought to apply.
Regardless of the excellent quality of the predictions made, for example, by
quantum field theory, the existence of the electron is not a
certainty.\footnote{%
   The electron might exist only as a mathematical model in the mind and not as
   a reality in itself.  I do not deny the possibility of the electron's
   existence in itself, but I \emph{do} deny that anyone could ever be certain
   of such existence. The best scientific theory 1,000 years from now might not
   refer to the present idea of the electron except as an approximation that is
   appropriate in certain contexts.
}
At some point in the future, predictions made on the basis of the idea of the
electron might fail to match what is perceived by the senses.  As Edward
Harrison points out in his {\it Cosmology}, a mask that hides a face should not
be mistaken for the face itself. A scientific theory and the sense perceptions
that it explains form a mask that covers nature's face. The amazing ability of
modern scientific theory, enabled by the calculus, to predict sense experience
across a wide range of conditions presents the strong temptation to mistake the
mask for the face, but the face will forever remain hidden.

My purpose here is to write a calculus textbook that, through careful
development of both the mathematics and some exemplifying aspects of the
mechanical theory, makes clear what the calculus is, how it can be applied in
the service of scientific theory, and what the limits of scientific theory are.

\vspace{0.25in}

\begin{flushright}
Thomas E. Vaughan

2014 August
\end{flushright}

